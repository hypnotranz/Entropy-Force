\documentclass{article}

\usepackage{geometry}
\usepackage{microtype}
\usepackage{array}
\usepackage[table]{xcolor}
\usepackage{mathtools}


\newcolumntype{P}{%
>{\rule[-0.6cm]{0pt}{1.5cm}\centering$}p{1cm}<{$}}

\newcommand{\tbf}[1]{\textbf{#1}}

\begin{document}

Say we have a $5\times5$ grid with a unit somewhere on it. We want to apply a causal entropic force to it and see what happens. Each turn, the unit may stand still or move in any of the available four cardinal directions.

Suppose the unit begins in the cell shaded blue.

\vspace{1em}

\noindent\begin{tabular}{|P|P|P|P|P|}
\hline
(1,1) & (1,2) & (1,3) & (1,4) & (1,5) \tabularnewline
\hline
(2,1) & \cellcolor{blue!25}(2,2) & (2,3) & (2,4) & (2,5) \tabularnewline
\hline
(3,1) & (3,2) & (3,3) & (3,4) & (3,5) \tabularnewline
\hline
(4,1) & (4,2) & (4,3) & (4,4) & (4,5) \tabularnewline
\hline
(5,1) & (5,2) & (5,3) & (5,4) & (5,5) \tabularnewline
\hline
\end{tabular}

\vspace{1em}

Since we will apply the function $\nabla S_c(\textbf{X},t)$, we need to figure out how to write $S_c(\textbf{X},t)$. The state of the system at any time, $\textbf{X}$, is the $x$ and $y$ positions of the unit so we can define $\textbf{X}\equiv (x,y)$.

\section{Greedy Algorithm}

The greedy version of CEF is when the program only looks one step into the future, $t=1$. For our initial condition, we have five possible actions the unit may take: it may move N,S,E,W, or stand still. This brings us to an integral that took me a while to figure out:

\begin{equation}
S_c(\textbf{X},t)=-\int_{\tbf x(t)} \Pr(\tbf x(t)|\tbf x(0))\ln\Pr(\tbf x(t)|\tbf x(0))\mathcal{D}\tbf x(t)
\end{equation}

Since we are dealing with discrete states and only looking one step into the future, this integral can be written as a sum.

\begin{equation}
=-\sum_{\tbf x(1)} \Pr(\tbf x(1)|\tbf x(0)=\tbf X)\ln\Pr(\tbf x(1)|\tbf x(0)=\tbf X)
\end{equation}

For the greedy algorithm we are only looking one step into the future ($t=1$), so there are only five states $\tbf x(1)$ such that $\Pr(\tbf x(1)|\tbf x(0)=\tbf X)\ne 0$, making the domain

\begin{align}
\tbf x(0) &= (2,2) = \tbf X\\
\tbf x(1) &\in\{(1,2),(2,1),(2,2),(2,3),(3,2)\}
\end{align}

\clearpage

The possible locations at $t=1$ are:

\vspace{1em}

\noindent\begin{tabular}{|P|P|P|P|P|}
\hline
(1,1) & \cellcolor{blue!25}(1,2) & (1,3) & (1,4) & (1,5) \tabularnewline
\hline
\cellcolor{blue!25}(2,1) & \cellcolor{blue!25}(2,2) & \cellcolor{blue!25}(2,3) & (2,4) & (2,5) \tabularnewline
\hline
(3,1) & \cellcolor{blue!25}(3,2) & (3,3) & (3,4) & (3,5) \tabularnewline
\hline
(4,1) & (4,2) & (4,3) & (4,4) & (4,5) \tabularnewline
\hline
(5,1) & (5,2) & (5,3) & (5,4) & (5,5) \tabularnewline
\hline
\end{tabular}

\vspace{1em}

For simplicity, we'll let each possible action be equally likely - in reality it is probably better to add weights to certain classes of actions. In our case, this makes the sum very simple:

\begin{equation}
=-\sum_{x(1)_i} \frac15 \ln\frac15
\end{equation}

As there are five possibilities, this simplifies to

\begin{equation}
=-5\cdot\frac15\ln\frac15 = -\ln\frac15 = \ln 5
\end{equation}

So the future path entropy of our unit at its current location is $\ln 5$. It is actually the case that if there are $n$ equally likely paths, the path entropy for one time step is $\ln n$.

\clearpage

We can do the same calculation for each cell on the grid, producing:

\vspace{1em}

\noindent\begin{tabular}{|P|P|P|P|P|}
\hline
\ln3 & \cellcolor{blue!25}\ln 4 & \ln 4 & \ln 4 & \ln 3 \tabularnewline
\hline
\cellcolor{blue!25}\ln 4 & \cellcolor{blue!25}\ln 5 & \cellcolor{blue!25}\ln 5 & \ln 5 & \ln 4 \tabularnewline
\hline
\ln 4 & \cellcolor{blue!25}\ln 5 & \ln 5 & \ln 5 & \ln 4  \tabularnewline
\hline
\ln 4 & \ln 5 & \ln 5 & \ln 5 & \ln 4 \tabularnewline
\hline
\ln 3 & \ln 4 & \ln 4 & \ln 4 & \ln 3 \tabularnewline
\hline
\end{tabular}

\vspace{1em}

So our unit will feel a tug towards the center:
\begin{equation}
\nabla S_c((2,2),1) = (\ln5-\ln4,\ln5-\ln4) \approx (0.4,0.4)
\end{equation}

\section{}

But the greedy algorithm only seeks to increase instantaneous path entropy with each step - it will not autonomously seek goals. For that, we need to get the program to see several steps into the future.

Na\"ively, we might try to count every possible path the system can take. This quickly becomes unmanageable. If you'd like the math proof for why the following works, let me know - it just takes a while to type up.

We can write $S_c(\tbf X,t)$ as a recursive function:

\begin{align}
S_c(\tbf X,t) &= S_c(\tbf X,1) + \sum_{\tbf x(1)} \Pr(\tbf x(1)|\tbf x(0)=\tbf X)S_c(\tbf x(1),t-1)\\
S_c(\tbf X,1) &= -\sum_{\tbf x(1)} \Pr(\tbf x(1)|\tbf x(0)=\tbf X)\ln \Pr(\tbf x(1)|\tbf x(0)=X)
\end{align}

So the path entropy for $t=2$ for our unit would become
\begin{equation}
\ln 5 + \frac15 \left( 2\ln4 + 3\ln5 \right)
\end{equation}
as there are two states the unit can move to with path entropy $\ln4$, two with path entropy $\ln5$, and if it stays still it has path entropy $\ln5$. Since the tile the unit is on has path entropy $\ln5$, that is added to the sum.

\clearpage

So to compute each iteration, all we need is:

\begin{itemize}
\item The states \tbf X can validly transition to (and the probabilities of doing so)
\item The path entropy for each state using the greedy algorithm
\end{itemize}

In my current projects, I implement this by using a lookup table - with each iteration I fill in a new layer.

\end{document}